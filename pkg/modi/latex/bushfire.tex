\inputencoding{latin1}
\HeaderA{bushfire}{Bushfire scars}{bushfire}
\methaliasA{bushfire.weights}{bushfire}{bushfire.weights}
\aliasA{bushfirem}{bushfire}{bushfirem}
\keyword{datasets}{bushfire}
%
\begin{Description}\relax
The bushfire data set was used by Campbell (1984, 1989) to locate bushfire scars. 
The dataset contains satelite measurements on five frequency bands, corresponding to each of 38 pixels. 
\end{Description}
%
\begin{Usage}
\begin{verbatim}
data(bushfire)
\end{verbatim}
\end{Usage}
%
\begin{Format}
A data frame with 38 observations on the 5 variables.
\end{Format}
%
\begin{Details}\relax
The data contains an outlying cluster of
observations 33 to 38 a second outlier cluster of observations
7 to 11 and a few more isolated outliers, namely
observations 12, 13, 31 and 32. bushfirem is created from bushfire by setting a proportion of 0.2 of the values to missing.
\end{Details}
%
\begin{Source}\relax
bushfirem:\code{
set.seed(234567891)
miss.rate <- 0.2
miss.ind<-rep(F,n*p)
miss.ind[sample(n*p,floor(miss.rate*n*p))]<-T
bushmiss<-matrix(miss.ind,ncol=5)
mean(bushmiss)
bushfirem<-bushfire
bushfirem[bushmiss]<-NA}

For testing purposes weights are provided: 
\code{bushfire.weights<-rep(c(1,2,5),length=nrow(bushfire))}

\end{Source}
%
\begin{References}\relax
Campbell, N. (1989) Bushfire mapping using noaa avhrr data. Technical Report. Commonwealth Scientific and
Industrial Research Organisation, North Ryde.
\end{References}
%
\begin{Examples}
\begin{ExampleCode}
data(bushfire)
## maybe str(bushfire) ; plot(bushfire) ...
\end{ExampleCode}
\end{Examples}
