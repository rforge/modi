\inputencoding{latin1}
\HeaderA{TRC}{Transformed rank correlations for multivariate outlier detection}{TRC}
\keyword{robust}{TRC}
\keyword{multivariate}{TRC}
\keyword{survey}{TRC}
%
\begin{Description}\relax
TRC starts from bivariate Spearman correlations and obtains a positive definite covariance matrix by 
back-transforming robust univariate medians and mads of the eigenspace. TRC can cope with missing values by a 
regression imputation using the a robust regression on the best predictor and it takes sampling weights into account.
\end{Description}
%
\begin{Usage}
\begin{verbatim}
TRC(data, weights, overlap = 3, mincor = 0, robust.regression = "rank", gamma = 0.5, prob.quantile = 0.75, alpha = 0.05, md.type = "m", monitor = F)
\end{verbatim}
\end{Usage}
%
\begin{Arguments}
\begin{ldescription}
\item[\code{data}] a data frame or matrix with the data
\item[\code{weights}] sampling weights
\item[\code{overlap}] minimum number of jointly observed values for calculating the rank correlation
\item[\code{mincor}] minimal absolute correlation to impute
\item[\code{robust.regression}] type of regression: "irls" is iteratively reweighted least squares M-estimator, "rank" is 
based on the rank correlations
\item[\code{gamma}] minimal number of jointly observed values to impute
\item[\code{prob.quantile}] if mads are 0 try this quantile of absolute deviations
\item[\code{alpha}] \code{(1-alpha)} Quantile of F-distribution is used for cut-off

\item[\code{md.type}] Type of Mahalanobis distance when missing values occur: "m" marginal (default), "c" conditional
\item[\code{monitor}] if \code{TRUE} verbose output
\end{ldescription}
\end{Arguments}
%
\begin{Details}\relax
TRC is similar to a one-step OGK estimator where the starting covariances are obtained from rank correlations and an 
ad hoc missing value imputation plus weighting is provided.
\end{Details}
%
\begin{Value}
The output of TRC is stored in two global variables \code{TRC.r} and \code{TRC.i}. \code{TRC.r} contains the following components:
\begin{ldescription}
\item[\code{sample.size }] number of observations
\item[\code{number.of.variables}] number of variables
\item[\code{number.of.missing.items}] number of missing values
\item[\code{significance.level}] \code{1-alpha}
\item[\code{computation.time}] elapsed computation time
\item[\code{medians}] componentwise medians
\item[\code{mads}] componentwise mads
\item[\code{center}] location estimate
\item[\code{scatter}] covariance estimate
\item[\code{robust.regression}] input parameter
\item[\code{md.type}] input parameter
\item[\code{cutpoint}] The threshold MD-value for the cut-off of outliers

\end{ldescription}
\code{TRC.i} contains two vectors of length \code{nrow(data)}:

\begin{ldescription}
\item[\code{outind}] Indicator of outliers
\item[\code{dist}] Mahalanobis distances (with missing values)
\end{ldescription}
\end{Value}
%
\begin{Author}\relax
 Beat Hulliger
\end{Author}
%
\begin{References}\relax
B\bsl{}'eguin, C., and Hulliger, B. (2004). Multivariate oulier detection in
incomplete survey data: The epidemic algorithm and transformed
rank correlations. Journal of the Royal Statistical Society, A
167(Part 2.), 275-294. 
\end{References}
%
\begin{Examples}
\begin{ExampleCode}
data(bushfirem,bushfire.weights)
TRC(bushfirem,weights=bushfire.weights)
\end{ExampleCode}
\end{Examples}
