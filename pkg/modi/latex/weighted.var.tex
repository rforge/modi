\inputencoding{latin1}
\HeaderA{weighted.var}{Weighted univariate variance coping with missing values}{weighted.var}
%
\begin{Description}\relax
This function is as weighted.mean. The squares are weighted with w and the divisor is sum(w)-1. 
\end{Description}
%
\begin{Usage}
\begin{verbatim}
weighted.var(x, w, na.rm = FALSE)
\end{verbatim}
\end{Usage}
%
\begin{Arguments}
\begin{ldescription}
\item[\code{x}] a vector with data
\item[\code{w}] positive weights
\item[\code{na.rm}] if \code{TRUE} remove missing values
\end{ldescription}
\end{Arguments}
%
\begin{Value}
\begin{ldescription}
\item[\code{comp1 }] Description of 'comp1'
\item[\code{comp2 }] Description of 'comp2'
\end{ldescription}
\end{Value}
%
\begin{Author}\relax
Beat Hulliger
\end{Author}
%
\begin{References}\relax
 \textasciitilde{}put references to the literature/web site here \textasciitilde{} 
\end{References}
%
\begin{SeeAlso}\relax
See Also as \code{\LinkA{weighted.mean}{weighted.mean}}
\end{SeeAlso}
%
\begin{Examples}
\begin{ExampleCode}
x<-rnorm(100)
x[sample(1:100,20)]<-NA
w<-rchisq(100,2)
weighted.var(x,w,na.rm=TRUE)
\end{ExampleCode}
\end{Examples}
